\documentclass[11pt]{article}

\usepackage{amsmath}
\usepackage{hyperref}
\usepackage{xepersian}

\author{پرهام الوانی}
\title{تکلیف معماری کامپیوتر سری ۶}
\begin{document}
\begin{titlepage}
\begin{center}
\emph{به نام خدا}
\end{center}
\maketitle
\begin{center}
\end{center}
\end{titlepage}
\tableofcontents
\newpage
\section{مساله ۱}
با فرض اینکه پایپلاین به جالت پایدار رسیده است برای قسمت اول داریم:
\begin{align*}
	\frac{150 * 10^6}{2} = 75 * 10^6 = 75 MIPS
\end{align*}
در ادامه برای قسمت دوم با توجه به اینکه پیش از واکشی عملوند‌‌ها باید منتظر اجرای دستور العمل قبلی باشیم پس 
هر دستور العمل به اندازه ۴ کلاک طول می‌کشد.
\begin{align*}
	\frac{150 * 10^6}{4} = 37.5 MIPS 
\end{align*}
\section{مساله ۲}
\section{مساله ۳}
ابتدا تعداد بیت های انتقالی در ثانیه را حساب میکنیم.
\begin{align*}
	data &= 2400 * 8 = 19200
\end{align*}
حال از آنحا که در هر بار ۱۶ بیت داده منتقل میشود پس نیاز به ۱۲۰۰ سیکل در هر ثانیه داریم.
\begin{align*}
	cycles &= 10^6 - 1200 = 998800
\end{align*}
\begin{align*}
	\frac{998800}{1000000} &= 99.88\% \\
	& 100 - 99.88 = 0.12\%
\end{align*}
\section{مساله ۴}
\begin{align*}
	t_{F_{1}} &= 180 * \frac{60}{100} = 108s \\
	t_{F_{2}} &= 108 * \frac{145}{100} = 156.60 \\
	t_{T_{2}} &= 156.56 + 72 = 228.56s
\end{align*}
\end{document}