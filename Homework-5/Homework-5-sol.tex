\documentclass[11pt]{article}

\usepackage{amsmath}
\usepackage{hyperref}
\usepackage{xepersian}

\author{پرهام الوانی}
\title{تکلیف معماری کامپیوتر سری ۵}
\begin{document}
\begin{titlepage}
\begin{center}
\emph{به نام خدا}
\end{center}
\maketitle
\begin{center}
\end{center}
\end{titlepage}
\tableofcontents
\newpage
\section{مساله ۱}
ابتدا عدد را به مبنای ۲ میبریم.
\begin{align*}
	-37.75 &= \\
	&= -100101.11 \\
	&= -1.0010111 * 2^{5}
\end{align*}
در ادامه عدد را به فرمت \lr{IEEE} باز نویسی میکنیم.
\begin{align*}
	Sign &= 1 \\
	Exponent &= 10000100 \\
	Fraction &= 00101110000000000000000 \\
	&\rightarrow 1100\_0010\_0001\_0111\_0000\_0000\_0000\_0000 \\
	&\rightarrow (C2170000)_{16}
\end{align*}
ابتدا عدد را در مبنای ۲ بازنویسی میکنیم.
\begin{align*}
	(40200000)_{16} &= 0100\_0000\_0010\_0000\_0000\_0000\_0000\_0000 \\
	Sign &= 0 \\
	Exponent &= 10000000 \\
	Fraction &= 01000000000000000000000
\end{align*}
حال عدد را در مبنای ۱۰ یازنویسی میکنیم.
\begin{align*}
	1.01 * 2^{1} &= \\
	&= 1.1 \\
	&= 1.5
\end{align*}

\section{مساله ۲}
\begin{align*}
	x_{min} &= -0.11111111 * 2_{4} &= -1111.1111 &= -15.9375 \\
	x_{max} &=  0.11111111 * 2_{4} &=  1111.1111 &=  15.9375
\end{align*}
\section{مساله ۳}
ابتدا دقت و محدوده اعداد را برای حالت اول به شرح زیر بدست میاوریم: (با فرض استاندارد \lr{IEEE})
\begin{align*}
	x_{min} &= - (2 - 2^{-12}) * 2^{2^{19} - 1} \\
	x_{max} &= (2 - 2^{-12}) * 2^{2^{19} - 1} \\
	Accuracy &= 2^{-12} * 2^{-2^{19}}
\end{align*}
و در ادامه برای حالت دوم داریم:
\begin{align*}
		x_{min} &= - (2 - 2^{-25}) * 2^{2^{6} - 1} \\
		x_{max} &= (2 - 2^{-25}) * 2^{2^{6} - 1} \\
		Accuracy &= 2^{-25} * 2^{-2^{6}}
\end{align*}
\end{document}